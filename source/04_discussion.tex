\section{Discussion and Conclusion}

In the present study we investigated the allocation of gaze while action execution and action planning with varying task complexity. We report four main findings with this study. Firstly, we found that when compared to a greedy depth-first search algorithm subjects did not perform optimally in sorting the objects. Compared to the EASY task, performance dropped considerably in the HARD task. Secondly, they used ad-hoc spatial heuristics such as a spatial bias for picking up and dropping off objects to reduce the search space resulting in moderately increased number of object displacements as compared to the optimal model. Thirdly, based on the scan path transitions, we observed greater net transitions in the EASY trials compared to HARD trials. The lower proportion of net transitions to and from the ROIs in HARD trials is evidence for "instability" in attention and heightened distraction. Thirdly, the relative net transitions in the action planning phase of both EASY and HARD trials negatively correlated with the number of object displacements made by the subjects. Finally, the relative timing of the first fixations on the immediate task-relevant ROIs in the action planning and execution phase revealed that subjects predominantly made just-in-time fixations before executing the pick-up and drop-off actions.In sum, our findings reveal a just-in-time strategy for problem solving which is reflected in both behavior and attention allocation and is invariant to task complexity.

To assess the sorting behavior of the participants we compared their object displacement behavior with a greedy depth-first search algorithm which optimizes for the shortest path to the solution. Studies in human performance in reasoning tasks as well as combinatorial optimization problems \citep{MacGregor2011-yf} have revealed that humans solve these tasks in a self-paced manner rather than being dependent on the size of the problem. \citet{Pizlo2005-qi} found that subjects do not perform an implicit search of the problem state space where they plan the moves without executing them, i.e. longer solution times would lead to shorter solution paths. Instead, they showed that humans break the problem down to component sub-tasks which gives rise to a low-complexity relationship between the problem size and time to solution. Further, \citet{Pizlo2005-qi} show that instead of using implicit search, subjects use simpler heuristics to decide the next move. To this effect, while subjects in our study are sub-optimal compared to a depth-first search algorithm, humans in general are prone to use non-complex heuristics that favor limited allocation of resources in the working memory.

We further modelled the gaze guidance behavior of the participants while they performed the sorting task. Based on the regression model, we can conclude that the subjects had higher gaze guidance for the EASY trials compared to the HARD trials. These results suggest that subjects' gaze moved to the target locations smoothly without much search process in the EASY trials. This suggests that owing to the task complexity of the HARD trials, subjects made many fixations on different objects as a way to select the target objects and shelves. Moreover, we also see higher gaze guidance in the planning epochs as compared to execution epochs. The higher F-values in the planning phase suggests that subjects subjects picked up objects without doing much search, which also indicates why they showed sub-optimal behavior while sorting the task. The higher F-values in the execution epochs can be interpreted in two ways. Firstly, subjects maybe picked up the objects at whim and then searched for the best way to place the object, indicating the actual planning of the action happens  after the action has already been initiated. Alternatively, the lower F-values would indicate that they moved their gaze to other objects as a way to monitor the outcome of the current task and produce the best outcome. We reject the second alternative, as subjects mostly show gaze transitions to other objects/shelf that are not relevant in the previous, current, next action sequence. If monitoring of the current action took place, subjects would have ideally made more fixations to the previous task relevant objects. Furthermore, our data also revealed that trials with greater number of object displacements showed lower F-values in the planning epochs in HARD trials, with F-values negatively correlated to the number of object displacements. Here, we argue that due to the complex nature of the HARD tasks, subjects fixated on many objects before selecting one for an action. As the subjects were more constrained in the HARD tasks they needed to search the scene for relevant objects more often. In sum, our findings show that the task complexity coupled with the situation of planning or executing the tasks had significant effects on gaze guidance behavior. 

We also studied the latency of the first fixations on the task-relevant regions of interest for the trial types and action epochs. Our findings show that subjects made fixations to the task-relevant object or shelf towards the latter half of the epochs. These findings suggest that subjects predominantly used a just-in-time strategy for selecting objects, i.e. they acted on the target objects closely after fixating on them. Interestingly, in the action execution epochs, subjects make first fixations on the target objects that are relevant to the next action in the sequence. These fixations can be categorized as look-ahead fixations as have been found by Pelz et al. (ref), Mennie and Hayhoe (ref) Mars and Navarro (ref). These studies showed that look-ahead fixations have a low-probability of occurrence at about 20\% of the reaching movements (ref). These fixations usually relate to anticipatory eye movements and are a task-dependent strategy to acquire information about objects for future manipulation.

Taken together, our findings are also consistent from the perspective of embodied cognition \citep{Wilson2002-mv, Ballard2013-lo, Van_der_Stigchel2020-qb}, where: a) we off-load cognitive work on the environment; b) the environment is central to the cognitive system; b) cognition is for action; c) we off-load cognitive work on the environment; d) cognition is situated. The behavioral results show that subjects use spatial heuristics to complete the task indicating they exploit the external world to reduce the cognitive effort of selecting optimal actions. The oculomotor behavior further suggests that the just-in-time fixation strategy is primarily for action initiation. Further, \citet{Droll2007-bp} have suggested that the just-in-time strategy to lower the cognitive cost of encoding objects in the world into the visual working memory. \citet{Konig2016-ew} have further proposed eye movements reveal much about the cognitive state. In our study, the differences in the gaze guidance behavior for the different trial types and action epochs further reveal that the cognitive processes are situated in the the current context of the environment and inherently involves perception and action.

In conclusion, in the present study we investigated the oculomotor responses to novel task scenarios that varied in complexity. The main findings of the study was the just-in-time strategy of fixations employed by subjects to complete the task and that are tightly coupled with the action sequences in the task. We also show that subjects use this strategy to offset the cognitive effort of optimally planning their actions. This study extends the previous research showing the oculomotor behavior during natural tasks of tea-making, sandwich-making, hand-washing, etc., that are over-learned. Due to the abstract nature of the task, we believe our work offers a view of eye movement behavior in natural environments that generalizes across tasks.



