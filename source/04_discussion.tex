\section{Discussion and Conclusion}

In the present study we investigated the spatio-temporal aspects of gaze control while action execution and action planning with varying task complexity. We report five main findings with this study. First, we found subjects used ad-hoc spatial heuristics to reduce the solution search space resulting in moderately increased number of object displacements as compared to the optimal model. Second, fixations locked to the action initiation, revealed a prevalence of look-ahead and task monitoring fixations in the HARD tasks but not in the EASY task. Third, based on the scan path transitions, we observed greater relative net transitions in the EASY trials compared to HARD trials. The lower proportion of net transitions to and from the ROIs in HARD trials is further evidence for an extensive search behavior and alternating action guiding and monitoring fixations. Fourth, the relative timing of the first fixations on the immediate task-relevant ROIs in the action planning and execution phase revealed a systematic sequence of fixations leading up to the immediate task-relevant ROI, indicating a just-in-time strategy of action supporting fixations. Finally, task complexity affected the systematic temporal sequence of fixations in the action planning phase but not in action execution phase. In sum, our findings reveal a structured  effect of task complexity on the spatio-temporal features of relevant action supporting cognitive schemas.

The central aim of the our study is to generalize the cognitive mechanisms of gaze control for tasks that are not over-learned and routine. Building on the work of \citet{Land1999-ol} where eye movements were studied while preparing tea, the tasks in our study are novel in the sense that they do not have an inherent action sequence attached to them. Our experimental setup provided a way to capture oculomotor behavior for tasks with varying complexity and that did not have a strict action sequence. By studying the dispersion of fixations on the previous, current, and next task relevant ROIs, we show a structured sequence of fixations that can be classified into look-ahead fixations, directing fixations, guiding fixations, checking fixations, etc. Our results generalize the occurrence of these fixations to the action sequence and the task complexity by doing away with object identity. 

\citet{Land1999-ol} proposed the various functions of eye movements from locating, directing, guiding, checking during tea-making. In our study, we also showed the occurrence of these fixations time-locked to the action initiation. Importantly, our study shows that task complexity affects the proportion and timing of fixations for locating/look-ahead as well as for checking the task progression. While look ahead fixations occurred predominantly before the initiation of the immediate action, the checking fixations occurred in parallel with the action execution. Interestingly, directing and guiding fixations were not affected by task complexity indicating that these fixations are central to the action repertoire.

\citet{Sullivan2021-ts} recently showed the various timescales at which look-ahead fixations can occur while assembling a tent. They reported look-ahead fixations attributed to the current sub-task and within 10 seconds of motor manipulation. In our study, we found look-ahead fixations on the upcoming  action related to the drop-off location of the current object displacement as well as towards the next object displacement sub-task. Our results show that these fixations are more salient in the more complex tasks. As evidenced by the relative net transitions to the task-relevant ROIs, there are more transitions in the action planning phase of the cognitively demanding tasks but they are not made at random. Further, we also show the systematic latency of these fixations to the immediate action.  Hence, the look-ahead fixations described in our study are part of an elaborate cognitive schema and less likely due to incidental fixations on the task based ROIs.

\citet{Land1997-ug} have further elaborated on the schemas that direct both eye movements and the motor actions. They proposed a chaining mechanism by which one action leads to another where the eyes supply constant information to the motor system via a buffer. In our analysis, the strict temporal structure of the first fixations on the ROIs lends further evidence of such a cognitive buffer that facilitates moving from one action to another where interspersed eye movements queue-in the objects and locations necessary for current and future actions. Importantly, the final first fixation is always the object or location necessary for the immediate sub-task, indicating the universality of the 'just-in-time' nature of these action-related fixations. Taken together, our study further corroborates the cognitive schemas that sequentially support action planning and execution.

Our results show higher task solutions with increasing task complexity. To assess the sorting behavior of the participants we compared their object displacement behavior with a greedy depth-first search algorithm which optimizes for the shortest path to the solution. Studies in human performance in reasoning tasks as well as combinatorial optimization problems \citep{MacGregor2011-yf} have revealed that humans solve these tasks in a self-paced manner rather than being dependent on the size of the problem. \citet{Pizlo2005-qi} found that subjects do not perform an implicit search of the problem state space where they plan the moves without executing them, where longer solution times would lead to shorter solution paths. Instead, they showed that humans break the problem down to component sub-tasks which gives rise to a low-complexity relationship between the problem size and time to solution. Further, \citet{Pizlo2005-qi} show that instead of using implicit search, subjects use simpler heuristics to decide the next move. To this effect, while subjects in our study are sub-optimal compared to a depth-first search algorithm and use an arbitrary spatial heuristics during the task, humans in general are prone to use non-complex heuristics that favor limited allocation of resources in the working memory. Hence, the higher time duration and number of object displacements shown by the participants do not necessarily demonstrate a lack of planning.

From the perspective of embodied cognition \citep{Wilson2002-mv, Ballard2013-lo, Van_der_Stigchel2020-qb}, humans off-load cognitive work on the environment and the environment is central to the cognitive system. The behavioral results show that subjects use spatial heuristics to complete the tasks indicating they exploit the external world to reduce the cognitive effort of selecting optimal actions. Further, \citet{Droll2007-bp} have suggested that the just-in-time strategy to lower the cognitive cost of encoding objects in the world into the visual working memory. In the embodied cognition framework, cognition is situated and for producing actions. With our study, we show that task complexity affected various spatio-temporal features of gaze control in the action planning and execution phase. These findings are also in line with \citet{Konig2016-ew} illustrating that eye movements reveal much about the cognitive state. 

In conclusion, in the present study we investigated the oculomotor responses to novel task scenarios that varied in complexity. Our results generalize past work on action-oriented eye movements to tasks that are not routine or over-learned such as tea-making, sandwich-making, hand-washing, etc. We show that eye movements support various functions of locating object of interest, directing and guiding the body and hands, as well as monitoring the task progression. Furthermore, we show how fixations are tightly coupled with the action sequences in the task. More importantly we show the prevalence of cognitive schemas that are affected by the task complexity even when sub-optimal decisions are made by the actor. 



