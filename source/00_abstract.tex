Eye movements in the natural environment have primarily been studied for over-learned everyday activities such as tea-making, sandwich making, driving that have a fixed sequence of actions associated with them. These studies indicate an interplay of low-level action schemas that facilitate task completion. However, it is unclear if this strategy is also in play when the task is novel and a sequence of actions must be planned in the moment. To study attention mechanisms in a novel task in a natural environment, we recorded gaze and body movement data in a virtual environment while subjects performed a sorting task where they sorted objects on a life-size shelf based on some object features. To study the action planning and execution related gaze guidance behavior we also controlled the complexity of the sorting task by introducing EASY and HARD tasks. We show that subjects are close to optimal while performing EASY trials and are more sub-optimal while performing HARD tasks. Fixations aligned with action onset show task complexity elicits greater proportion of look-ahead, and monitoring fixations but not directing and guiding fixations. Task complexity affected the scan-paths on the task-relevant regions of interest during action planning and execution where subjects exhibit a greater search and action monitoring behaviors in HARD tasks and less so in EASY tasks. Task complexity also affected the temporal sequence of first fixations on the task-relevant regions of interest systematically for action planning but not for action execution. Our findings show that task complexity modulates the competition of low-level cognitive schemas during planning and execution even when sub-optimal decisions are made by the actor.